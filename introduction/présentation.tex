\documentclass{article}

\usepackage[utf8]{inputenc}
\usepackage[T1]{fontenc}
\usepackage[french]{babel}

\usepackage{caption}
%\usepackage{pgfplots}
\usepackage{listings}
\usepackage{graphicx}
\usepackage{footnote}
\usepackage{amsmath}
\usepackage{amsthm}
\usepackage{graphicx}
\usepackage{url}
\usepackage{amssymb}
\usepackage{mathrsfs}
\usepackage{multirow}
\usepackage{amsfonts}
\usepackage[boxed,linesnumbered,noend]{algorithm2e}
\usepackage{qcircuit}
\usepackage{enumerate}

\newtheorem{thm}{Theorem}
\newtheorem{prop}{Propriety}
\newtheorem{lemma}{Lemma}
\newtheorem{defi}{Definition}
\newtheorem{coro}{Corollary}



\setlength{\oddsidemargin}{0pt}
% Marge gauche sur pages impaires
\setlength{\evensidemargin}{0pt}
% Marge gauche sur pages paires
\setlength{\textwidth}{470pt}
% Largeur de la zone de texte 
\setlength{\topmargin}{0pt}
% Pas de marge en haut
\setlength{\headheight}{13pt}
% Haut de page
\setlength{\headsep}{10pt}
% Entre le haut de page et le texte
\setlength{\footskip}{40pt}
% Bas de page + séparation
\setlength{\textheight}{630pt}
% Hauteur de la zone de texte

\title{Réunion de présentation Master CHPS}
\author{UVSQ}
\date{}

\newcommand{\note}{\medskip\noindent\underline}

\begin{document}
\maketitle


Le Master CHPS est le premier master HPC, qui commence même à être copié en province $\to$ fortes possibilités d'emploi et de thèses (sous réserve de résultats).
\bigskip

Attention, certains élèves ont un manque de prog parallèle (voir resources par mail). Ces étudiants devaient être placés dans un parcours simulation qui ne s'est pas ouvert (donc l'intégralité des élèves sera placée en filière Haute Performance).

\paragraph{Intitulé légal} Master Calcul Haute Performance, Simulation ; Option B2 Informatique Haute Performance \& Simulation.
\bigskip

\paragraph{Tronc commun + B2 :} 36 ECTS
\paragraph{Stage :} 24 ECTS
\bigskip

Le stage dure entre 5 et 6 mois ; le tronc commune \emph{comme les options} sont obligatoires (il y a 5 options à choisir parmi 5). Tous les cours sont à 3 ECTS et durent 20h de CM + 15h de TD/TP (sauf anglais).

L'emploi du temps se décompose en trois blocs séparés par deux semaines de révisions et une semaine de contrôles (partiels).

Les thèses CIFRE sont pratiques pour l'insertion professionnelle dans l'industrie par la suite.

Le cour d'anglais aura lieu le mercredi à partir du 27 septembre, durent 27h théoriquement mais ne propose actuellement pas ce volume horaire, ce thème sera abordé plus tard.

\paragraph{17 octobre :} Forum d'ORAP, journée dédiée à un aspect du HPC (Rien à faire à l'université ce jour-là, l'inscription est gratuite et obligatoire $\to$ en profiter pour remplir son carnet d'adresse...).

\paragraph{Obtention du Master :} les UE $>7$ sont compensables sous conditions que le tronc commun et le bloc option soit validé avec une moyenne $>10$ (pondérée par les ECTS... qui sont tous identiques....). La note de stage doit également être $>10$ et est pondéré par sont nombre d'ECTS pour la note finale.


\paragraph{Stage :} Commence en mars, la convention s'effectue via la DRIP \url{http://www.drip.uvsq.fr/}. Les bons sujet se trouvent au début (décembre-novembre $\to$ trop tard !). Cette année, les soutenances se sont effectuées le 21-22 septembre. Les stages ne doivent \emph{pas} dépasser le 30 septembre cependant.

\paragraph{Évaluation :} basée sur le contrôle continu (le retour des interros surprises...) $\to$ présence obligatoire !

\paragraph{Gestion du master :} Fabien Chevalier $\to$ Intermédiaire entre étudiants/enseignants et administration ; Charlotte Lemaire pour la gestion administrative. Contact au \url{prénom.nom@uvsq.fr}.

\paragraph{Contact responsable de la promotion :}\url{tassaditcilia@gmail.com} téléphone 0612731911 Ait Kaci Célia

\paragraph{Petite présentation des différents modules}

\end{document}