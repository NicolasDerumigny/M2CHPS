\documentclass{article}

\usepackage[utf8]{inputenc}
\usepackage[T1]{fontenc}
\usepackage[french]{babel}

\usepackage{caption}
%\usepackage{pgfplots}
\usepackage{listings}
\usepackage{graphicx}
\usepackage{footnote}
\usepackage{amsmath}
\usepackage{amsthm}
\usepackage{graphicx}
\usepackage{url}
\usepackage{amssymb}
\usepackage{mathrsfs}
\usepackage{multirow}
\usepackage{amsfonts}
\usepackage[boxed,linesnumbered,noend]{algorithm2e}
\usepackage{qcircuit}
\usepackage{enumerate}
\usepackage{eurosym}

\newtheorem{thm}{Théorème}
\newtheorem{prop}{Propriété}
\newtheorem{lemma}{Lemme}
\newtheorem{defi}{Définition}
\newtheorem{coro}{Corollaire}

\SetKwBlock{Label}{}{}
\SetKwRepeat{Do}{do}{while}
\SetKwBlock{Void}{void}{}
\SetKwBlock{Struct}{struct}{}

\setlength{\oddsidemargin}{0pt}
% Marge gauche sur pages impaires
\setlength{\evensidemargin}{0pt}
% Marge gauche sur pages paires
\setlength{\textwidth}{470pt}
% Largeur de la zone de texte 
\setlength{\topmargin}{0pt}
% Pas de marge en haut
\setlength{\headheight}{13pt}
% Haut de page
\setlength{\headsep}{10pt}
% Entre le haut de page et le texte
\setlength{\footskip}{40pt}
% Bas de page + séparation
\setlength{\textheight}{630pt}
% Hauteur de la zone de texte

\title{Evaluation de performances - TD2}
\author{Nicolas Derumigny}
\date{}

\newcommand{\note}{\medskip\noindent\underline}

\begin{document}

\paragraph{Ex 1:}
\begin{enumerate}
\item On trouve $40\%$ d'écritures pour $50\%$ de lectures

\item Les accès sont courts : pas plus de 15 blocs par requête, et une moyenne de 8 blocs par requêtes. La moyenne reste aux alentours de 8 : 7,985 pour les écritures et 8,004 pour les lectures. Le coefficient de variation vaut 0.54 dans les deux cas.

\item Les tailles des requêtes sont équidistribuées sur toues les valeurs entières entre 1 et 15

\item L'écart interquartile vaut 8, ce qui est logique avec le graph des décilles ($12-4$).

\item Le mode est la valeur la plus présente dans la série de donnée, elle vaut 1 pour les lecture, c'est loin de la moyenne ce qui la fait légèrement descendre

\item 


\end{enumerate}



\end{document}